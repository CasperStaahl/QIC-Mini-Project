\section{Hardware Execution}\label{sec:hardware-execution}
We have ensured that the algorithm can also be executed on real quantum hardware.
Through Microsoft Azure's workspace, we have gained access to hardware from different vendors.
The idea was to test the algorithm on one of these machines, but it turned out that the queue times for these quantum computers would be too long.
The reason for this is that every time a new iteration of the algorithm is run, a new separate query to the hardware queue has to be made.
This means that our test would go back in the queue for every instance of the algorithm, which we concluded was not viable with the given time.
Further, we were burning through our free credits, almost using our \$500 in just a couple of runs of the algorithm.

It should be noted that we queried small executions to the hardware provided by Quantinuum, which has provided us with several successful runs of the algorithm.

Nevertheless, we have also done test executions on Rigetti's free simulator.
Note that Quantinuum does provide a simulator for their platform, but this one also requires payment, and therefore Rigetti's simulator was chosen.
The results can be seen in \autoref{table:hardware-sim-results}.

\begin{table}[h!]
\centering
\begin{tabularx}{\textwidth}{|X|X|X|X|X|}
\hline
Max \# of Atoms & \# Connectivities & \# Tests &  \# Wrong 1st Run & \# Wrong 2nd Run \\
\hline
3 & 3 & 1000 & 19 & 4 \\
5 & 5 &  100 &  3 & 0 \\
\hline
\end{tabularx}
\caption{Hardware simulation results (Rigetti)}
\label{table:hardware-sim-results}
\end{table}
