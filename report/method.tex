\section{Method}\label{sec:method}

The main design objective is to create an algorithm that can solve the SAT problem using Grover's algorithm.
Further, the algorithm must be correct, and must be able to be both simulated and ran on real hardware.
In terms of constraints, we will not try to optimize the running time of the algorithm, and therefor we will not deal with parallelism and such.
The prequel was decided because the running time of the algorithm, both in simulation and on real hardware, would not be indicative of the possible speedup introduced by Grover's algorithm.
In particular, the simulation of a quantum circuit is much slower that the real execution, and to run quantum circuits on real quantum hardware one has to sit in hour long queues.

The algorithm in \autoref{fig:satisfiable-pseudocode} serves as a high-level overview, and hopefully convinces you that the solution will work.

\begin{figure}
\centering
\begin{minted}{python}
def satisfiable(p: Proposition) -> bool:
oracle = phase_oracle(p)
N = 2 ** count_atomic_propositions(p)
m = 1
while m <= sqrt(N):
    k = random{1, 2, ..., round(m)}
    grover_k_times = grover(oracle, k)
    assignment = grover_k_times.run()
    if valuation(assignment):
        return True
    m = (5 / 4) * m
return False
\end{minted}
\caption{High level pseudocode of \mintinline{python}{satisfiable} }
\label{fig:satisfiable-pseudocode}
\end{figure}

The idea is to convert \texttt{p} to an quantum circuit corresponding to the oracle in Grover's algorithm, and run Grover's algorithm  on this oracle until the probability of us missing a solution is sufficiently small.
For this purpose we use amplitude amplification:
we iteratively pick \texttt{k}, the number of Grover's operator applications from the integer range $[1, \operatorname{round}(\texttt{m})]$, with $\texttt{m} = 1$ initially.
If Grover's algorithm returns an assignment with the evaluates to true for the proposition \texttt{p} we stop, otherwise we continue increasing $\texttt{m}$ exponentially (in our case \mintinline{python}{m = (5 / 4) * m}) after each iteration.
We stop when $\texttt{m} \leq \sqrt{\texttt{N}}$.
