\section{State of the Art}\label{sec:state-of-the-art}

Song gives a nice outline of Grover's quantum search algorithm and the evolutions that followed in the early days~\cite{song2017early}.

Grover's algorithm is a promising unstructured search algorithm that takes advantage of our understanding of quantum computers.
When searching for a single solution in a search space of size $N$, the algorithm is proven to be asymptotically optimal with a worst case of $O\left(\sqrt{N}\right)$, making it quadratically faster than classical computing~\cite{grover1996fast}.
An outline of the algorithm can be found in \autoref{alg:grover-known}, please refer to \cite{song2017early} for more thorough explanation.

\begin{algorithm}
\caption{Grover's quantum search algorithm (number of solutions known)~\cite{song2017early}}
\begin{algorithmic}[1]
\State \textbf{Input:} $O_f$ with $f(x) = 1$ iff $x \in A$.
\State \textbf{Output:} $x \in A$, a marked item.
\State \textbf{Initialization:} $\lvert h \rangle := H^{\otimes n} \lvert 0^n \rangle$.
\State \textbf{Iteration:} apply $G = (-H^{\otimes n} Z_0 H^{\otimes n}) Z_f$ on $\lvert h \rangle$ $k$ times ($k$ depends on the number of solutions).
\State Measure and obtain candidate solution $x$.
\end{algorithmic}
\end{algorithm}\label{alg:grover-known}

In our particular case, we are working with a problem where there might be more than one solution, and the original algorithm proposed by Grover only supports a setting where one solution or marked items are present.
Luckily, Boyer et al. extended Grover's algorithm to support multiple marked items for both know and unknown amounts~\cite{boyer1998tight}.
In particular, Boyer et al. showed that the algorithm shown in \ref{alg:grover-unknown} will find a marked item in $O(\sqrt{N/a})$ iterations where $N$ is the size of the search space and $a$ the number of marked items.

\begin{algorithm}
\caption{Quantum search with number of solutions unknown~\cite{song2017early}}
\begin{algorithmic}[1]
\State \textbf{Input:} $O_f$ with $f(x) = 1$ iff $x \in A$. $\lambda = 6/5$.
\State \textbf{Output:} $x \in A$, a marked item.
\State Initialize $m = 1$.
\While{$m \leq \sqrt{N}$}
    \State pick uniformly random $k \leftarrow \{1, \ldots, m\}$.
    \State apply $k$ times the basic Grover iteration $G$ on initial state $\lvert h \rangle$.
    \State measure and obtain $x$. If $x \in A$, output $x$ and abort. Otherwise set $m \leftarrow \lambda m$.
\EndWhile
\end{algorithmic}
\end{algorithm}\label{alg:grover-unknown}
